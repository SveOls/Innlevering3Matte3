\section{Oppgave 3.3}

\subsection*{a)}

Dette er ekvivalent med matrisen 

\[
    \left[
    \begin{array}{cc|c}
        1&4&1\\
        5&18&8\\
        -3&4&2
    \end{array}
    \right] = 
    \left[
        \begin{array}{cc|c}
            1&0&-1/4\\
            0&1&5/16\\
            5&18&8
        \end{array}
    \right]
\]

Nei, det er ingen tall som gir rett svar for alle tre verdiene. 

\subsection*{b)}

\[
    \left[
    \begin{array}{ccc|c}
        1&4&a&d\\
        5&18&b&e\\
        -3&4&c&f
    \end{array}
    \right] = 
    \left[
        \begin{array}{ccc|c}
            1&0&\tfrac{a-c}{4}&x_a\\
            0&1&\tfrac{4b + 5(c-a)}{72}&x_b\\
            0&0&c&x_c
        \end{array}
    \right]
\]

Siste likhet er gitt ved at \(\tfrac{a-c}{4} = \tfrac{4b-5(c-a)}{72} = 0\). Da er \(a = c = 0\) og \(b = 0\), som gir polynomet \(x^2+1\).