\documentclass[pdftex, 10pt, norsk, a4paper, twoside]{article} 
\usepackage[top = 2.75cm, bottom = 4.50cm, left = 2.50cm, right = 2.50cm]{geometry}				
\usepackage[norsk]{babel}

\selectlanguage{norsk}

\usepackage[T1]{fontenc}

\usepackage[utf8]{inputenc}

\usepackage{graphicx}

\usepackage[section]{placeins}

\usepackage{cancel}
\usepackage{fmtcount}
\usepackage{siunitx} 
\usepackage{booktabs}

\usepackage{multirow}% Lar deg lage rader inni radene i en tabell.
\usepackage[pdftex,bookmarks,breaklinks]{hyperref}
\usepackage{parskip}% Gjør at latex tolker paragrafer og ikke bare ignorerer "white space"
\usepackage{rotating} %Gir deg en rekke muligheter for å rotere ting
\usepackage{amsmath , amsfonts , amssymb}							
\usepackage[hang,footnotesize,bf]{caption}% Gjør at captions under figurer og over tabeller blir pene.						
%\usepackage[version=3]{mhchem} % Tillater deg å skrive kjemiske formler
\usepackage{epstopdf}% Omgjør .eps-filer til .pdf filer. Denne pakken er genial når man bruker chemdraw der man kan eksportere som eps. 
\usepackage{textcomp,gensymb}
\usepackage{float}% Tillater å bruke stor H i floats, dette kalles ofte for float of doom og er skjeldent anbefalt.
\usepackage[super,square]{natbib}% Denne pakken er for bibliografien, super og square gjør at sitatet opphøyen og inlemmes av firkantparanteser. 
\usepackage{fancyhdr}% Lar deg lage den fine headeren på toppen av siden.
\usepackage{chemfig}
\usepackage{color}
\usepackage{mathtools}
\usepackage{placeins}
\DeclareSIUnit\atmosphere{atm}
\DeclareSIUnit\mmmercury{torr}
\DeclareSIUnit\mercury{Hg}
\DeclareSIUnit\molar{M}
\setcounter{tocdepth}{4}
\setcounter{secnumdepth}{4}
\newcommand*{\plimsoll}{{\ensuremath{-\kern-4pt{\ominus}\kern-4pt-}}}



\pagestyle{fancy}
\newcommand\EatDot[1]{} % Fjerner punktum på slutten av referansen.

%%%%%%%%%%%%%%%%%%%%%%%%%%%%%%%%%%%%%%%%%%%%%%%%%%%%%%%%%%%%%%%%%%%%%%


\makeatletter
\def\@seccntformat#1{%
  \expandafter\ifx\csname c@#1\endcsname\c@section\else
  \csname the#1\endcsname\quad
  \fi}
\makeatother


%%%%%%%%%%%%%%%%%%%%%%%%%%%%%%%%%%%%%%%%%%%%%%%%%%%%%%%%%%%%%%%%%%%%%%
%Setter noen instillinger for å gjøre ting pent :) 
\setlength\headheight{30pt}	
\floatstyle{plaintop}													
\restylefloat{table}													
\floatstyle{plain}														
\restylefloat{figure}													
\numberwithin{equation}{section}
\numberwithin{figure}{section}
\numberwithin{table}{section}
%%%%%%%%%%%%%%%%%%%%%%%%%%%%%%%%%%%%%%%%%%%%%%%%%%%%%%%%%%%%%%%%%%%%%%


%%%%%%%%%%%%%%%%%%%%%%%%%%%%%%%%%%%%%%%%%%%%%%%%%%%%%%%%%%%%%%%%%%%%%%
%Lager headeren
\lhead{KJ1041 \\ Sverre Olsen}			
\chead{}																
\rhead{\today}	
\lfoot{}
\cfoot{\thepage}												
\rfoot{}
%%%%%%%%%%%%%%%%%%%%%%%%%%%%%%%%%%%%%%%%%%%%%%%%%%%%%%%%%%%%%%%%%%%%%%
%Denne kodebiten lager en egen funksjon som heter \signature{}{} med input sted og navn. Den brukes litt lengre ned i koden
\newcommand{\signature}[2]{
\begin{minipage}[t]{0.9\textwidth}
\vspace{1cm}

    #1, \today
\vspace{1.5cm}

\noindent
\begin{tabular}{cc}
    \rule{6cm}{1pt} & \hspace{2cm} \\
    #2 & 
\end{tabular}
\vspace{1cm}
\end{minipage}
}
%%%%%%%%%%%%%%%%%%%%%%%%%%%%%%%%%%%%%%%%%%%%%%%%%%%%%%%%%%%%%%%%%%%%%%

\newcommand{\brk}{\,|\,}
\newcommand{\intinf}{\int_{-\infty}^\infty}
\newcommand{\expo}[1]{e^{#1}}
\newcommand{\braket}[1]{\ensuremath{\langle\,#1\,\rangle}}
\newcommand{\der}[1]{\ensuremath{\,\text{d}#1}}
\DeclareMathSymbol{@}{\mathord}{letters}{"3B}

\begin{document}

% \input{dokumenter/rapport} importerer fila som ligger i mappen Dokumenter og henter latexkoden derfra. Dette er for å ikke blande tekst og masse kode som er i dette dokumentet
\begin{center}
\LARGE{\textbf{KJ1041 - Øving 2}}
\end{center}
\bigskip


\bigskip
\bigskip
\bigskip

\section{Oppgave 3.1}

\subsection*{a)}

\[
\left[
\begin{array}{ccccccc}
    0&1&1&0&0&0&1 \\
    0&0&0&0&1&0&1 \\
    0&0&0&0&0&1&1 \\
    0&0&0&0&0&0&0
\end{array}
\right] \cdot
\left[
\begin{array}{c}
    a  \\
    b  \\
    c  \\
    d  \\
    e  \\
    f  \\
    g
\end{array}
\right] =
\left[
\begin{array}{c}
    0  \\
    0  \\
    0  \\
    0 
\end{array}
\right]
\]

\(
a, b, c, d, e, f, g = 0
\) er en gyldig løsning.

\subsection*{b)}

\[
    \left[
        \begin{array}{ccc}
            8&-7&0\\
            -8&-7&3\\
            -4&5&-8\\
            -6&6&-4
        \end{array}
    \right] \cdot
    \left[
        \begin{array}{c}
            a\\
            b\\
            c
        \end{array}
    \right] =
    \left[
        \begin{array}{c}
            -3\\
            -7\\
            -3\\
            0
        \end{array}
    \right]
\]

\[
    8a -7b = 3\quad-8a-7b+3c=-7\quad-4a+5b-8c=-3\quad-6a+6b-4c=0    
\]

\[
    \left[
        \begin{array}{ccc|c}
            8&-7&0&3 \\
            -8&-7&3&-7 \\
            -4&5&-8&-3 \\
            -6&6&-4&0
        \end{array}
    \right] = 
    \left[
        \begin{array}{ccc|c}
            1&0&0&\frac{83}{215} \\
            &&&\\
            0&1&0&\frac{187}{215} \\
            &&&\\
            0&0&1&\frac{156}{215} \\
            &&&\\
            0&0&0&0 \\
            
        \end{array}    
    \right]
\]

\bigskip

\section{Oppgave 3.2}
 


\bigskip

\section{Oppgave 3.3}

\subsection*{a)}

Dette er ekvivalent med matrisen 

\[
    \left[
    \begin{array}{cc|c}
        1&4&1\\
        5&18&8\\
        -3&4&2
    \end{array}
    \right] = 
    \left[
        \begin{array}{cc|c}
            1&0&-1/4\\
            0&1&5/16\\
            5&18&8
        \end{array}
    \right]
\]

Nei, det er ingen tall som gir rett svar for alle tre verdiene. 

\subsection*{b)}

\[
    \left[
    \begin{array}{ccc|c}
        1&4&a&d\\
        5&18&b&e\\
        -3&4&c&f
    \end{array}
    \right] = 
    \left[
        \begin{array}{ccc|c}
            1&0&\tfrac{a-c}{4}&x_a\\
            0&1&\tfrac{4b + 5(c-a)}{72}&x_b\\
            0&0&c&x_c
        \end{array}
    \right]
\]

Siste likhet er gitt ved at \(\tfrac{a-c}{4} = \tfrac{4b-5(c-a)}{72} = 0\). Da er \(a = c = 0\) og \(b = 0\), som gir polynomet \(x^2+1\).

\bigskip

\section{Oppgave 4.1}

\[
    \left[
    \begin{array}{cc}
        a&b\\
        c&d
    \end{array}    
    \right] \cdot
    \left[
        \begin{array}{c}
            e \\
            f
        \end{array}
    \right] = 
    \left[
        \begin{array}{c}
            1 \\
            -1
        \end{array}    
    \right]
\]

\[
    \left[
    \begin{array}{cc}
        a&b\\
        c&d
    \end{array}    
    \right] \cdot
    \left[
        \begin{array}{c}
            g \\
            h
        \end{array}
    \right] = 
    \left[
        \begin{array}{c}
            2 \\
            3
        \end{array}    
    \right]
\]


\[
    ae + bf = 1\qquad ce + df = -1 \qquad ag + bh = 2 \qquad cg + dh = 3
\]

\[
    \left[
    \begin{array}{cc}
        a&b\\
        c&d
    \end{array}    
    \right] \cdot
    \left[
    \begin{array}{c}
        2e-g\\
        2f-h
    \end{array}    
    \right]      
\]

\[
    2ae - ag + 2bf - bh = 2(ae + bf) - (ag + bh) = 0 
\]

\[
    2ce - cg + 2df - dh = 2(ce + df) - (cg + dh) = -5    
\]

\[
    A\textbf{w} = \left[
        \begin{array}{c}
            0 \\
            -5
        \end{array}    
    \right]
\]

\bigskip

\section*{Oppgave 4.2}

\subsection*{a)}



%\vfill % Gjør at signaturen havner nederst på siden, uavhengig av hvor mye tekst det er på den. Dette er en god praksis
%\signature{Trondheim}{Olsen S} % lager en signaturlinje.

\newpage 

\end{document} 